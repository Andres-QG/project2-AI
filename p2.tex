\documentclass[conference]{IEEEtran}

\hyphenation{op-tical net-works semi-conduc-tor}

\begin{document}

\title{Exploring Binary Classification in Supervised Learning: Comparative Analysis of Red Wine Quality and Mushroom Edibility Using ML Algorithms}

\author{\IEEEauthorblockN{Randy Agüero Bermúdez}
\IEEEauthorblockA{\textit{Escuela de Ciencias de la Computación e Informática} \\
\textit{Universidad de Costa Rica}\\
San José, Costa Rica \\
randy.aguero@ucr.ac.cr}
\and
\IEEEauthorblockN{Sara Espinoza Hernández.}
\IEEEauthorblockA{\textit{Escuela de Ciencias de la Computación e Informática} \\
\textit{Universidad de Costa Rica}\\
San José, Costa Rica \\
sara.espinoza@ucr.ac.cr}
\and
\IEEEauthorblockN{Andrés Quesada González}
\IEEEauthorblockA{\textit{Escuela de Ciencias de la Computación e Informática} \\
\textit{Universidad de Costa Rica}\\
San José, Costa Rica \\
andres.quesadagonzalez@ucr.ac.cr}
\and
\IEEEauthorblockN{Queene Zavala Morales.}
\IEEEauthorblockA{\textit{Escuela de Ciencias de la Computación e Informática} \\
\textit{Universidad de Costa Rica}\\
San José, Costa Rica \\
queene.zavala@ucr.ac.cr}
}

\maketitle

\begin{abstract}
  The present study explores the application of supervised learning approaches to binary classification problems, taking into consideration two different datasets: Red Wine Quality and the Mushroom Dataset. The effort involves data preprocessing, feature selection, and the application of four different machine learning algorithms, namely Logistic Regression, Decision Trees, k-Nearest Neighbors (kNN), and Neural Networks. These algorithms' performance evaluations are done via hyperparameter optimization and iteration of experiments with the help of various metrics such as accuracy, precision, recall, and ROC-AUC. The results will contain the fundamental relations of the tradeoffs that exist between model complexity and generalization, with computational efficiency regarding the chosen datasets. To be more specific, in the Red Wine dataset, wines are labeled as good or bad according to their physicochemical properties, while the Mushroom dataset involves classifying mushrooms into edible and toxic groups. The findings emphasize feature selection, dataset characteristics, and algorithm selection for achieving optimum classification performance. over-fitting and under-fitting, as well as the reproducibility of the results across experiments, are discussed in this paper, which comprehensively evaluates various methodologies related to supervised learning.
\end{abstract}

\begin{IEEEkeywords}
  supervised learning, binary classification, wine, mushrooms, machine learning algorithms, logistic regression, decision trees, k-nearest neighbors, neural networks, model evaluation metrics, machine learning, ai.
  \end{IEEEkeywords}

\section{Introduction}
Texto inicial.*

\subsection{Background}
Introduce el problema de clasificación binaria y su importancia en Machine Learning.

\subsection{Objective}
Explica los objetivos específicos del proyecto, como la comparación de algoritmos y el análisis de datasets.

\subsection{Contributions}
Resalta las contribuciones principales del paper, como el análisis comparativo y las conclusiones obtenidas.

\section{Methodology}

\subsection{Datasets}

\subsection{Data Preprocessing}

\subsection{Feature selection}

\subsection{Machine Learning Algorithms}
\subsubsection{Logistic Regression}
\subsubsection{Decision Trees}
\subsubsection{k-Nearest Neighbors}
\subsubsection{Neural Networks}

\subsection{Hyperparameter Optimization}

\subsection{Experimental Setup}
\subsection{Model Evaluation Metrics}


\section{Results and Discussion}
\subsection{Red Wine Quality Dataset}
\subsubsection{Logistic Regression}
\subsubsection{Decision Trees}
\subsubsection{k-Nearest Neighbors}
\subsubsection{Neural Networks}

\subsection{Mushroom Dataset}
\subsubsection{Logistic Regression}
\subsubsection{Decision Trees}
\subsubsection{k-Nearest Neighbors}
\subsubsection{Neural Networks}

\subsection{General Discussion}

\section{Conclusion}
The conclusion goes here.

\nocite{*}
\def\BibTeX{BibTeX}
\bibliographystyle{IEEEtran}
\bibliography{IEEEabrv,./refs}

\end{document}
